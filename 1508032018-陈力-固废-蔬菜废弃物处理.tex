% Created 2019-01-16 Wed 07:53
% Intended LaTeX compiler: xelatex
\documentclass[11pt]{article}
\usepackage{graphicx}
\usepackage{grffile}
\usepackage{longtable}
\usepackage{wrapfig}
\usepackage{rotating}
\usepackage[normalem]{ulem}
\usepackage{amsmath}
\usepackage{textcomp}
\usepackage{amssymb}
\usepackage{capt-of}
\usepackage{hyperref}
\usepackage{float}
\usepackage[UTF8]{ctex}
\setCJKmainfont{Sarasa Mono T CL}
\date{\today}
\title{1508032018-陈力-固废-蔬菜废弃物处理-文献宗述}
\hypersetup{
 pdfauthor={},
 pdftitle={1508032018-陈力-固废-蔬菜废弃物处理-文献宗述},
 pdfkeywords={},
 pdfsubject={},
 pdfcreator={Emacs 26.1 (Org mode 9.2)}, 
 pdflang={English}}
\begin{document}

\maketitle
\tableofcontents



\section{绪论}
\label{sec:org5d1832f}
\subsection{固体废物}
\label{sec:orge9c7deb}
固体废物主要在传统市场中产生,包括水果和蔬菜废物,这些废物主要在+市政土地填埋场
或倾倒场所处置。由于它们的性质和+组成,它们会相当快地降解并导致污垢+气味。在一些
水果和蔬菜废物(FVW)和食物+废物(FW)样品中,挥发性固体(VS)含量为80%±90%,
水含量为75%-95%。在城市固体废物(MSW)的收集,运输和土地填埋过程中,高浓度的有
机物和水含量是造成重气味+和大量渗滤液的主要原因。考虑到+高水分和有机物含量,这些
废物在生物处理(如厌氧消化)中比其他技术(如焚烧或堆肥)更有效地处理\cite{sitorus13_biogas_recov_from_anaer_diges}

\subsection{厌氧消化}
\label{sec:org4a37ae4}
厌氧消化是一种生物化学过程,其中微生物在没有氧气的情况下将生物可降解材料分解成生
物气(甲烷和二氧化碳)\cite{84_biogas_produc_utiliz} 。 那些生化过程非常复杂,难以在最佳条件下运行,因为必须
考虑许多参数并进行+控制\cite{deublein10_biogas_waste_renew_resour} 。 另外,由于技术或经济上的限制,一些参数难以估计,即,基板
消耗测量是昂贵的,需要三个小时并且离线完成\cite{carlos-hernandez09_fuzzy_obser_anaer_wwtp} 。在这
方面,数学+电子建模和计算机模拟是一个很好的工具。

\subsection{数学模型}
\label{sec:org43a93e6}
各种数学模型已经构建了用于+厌氧消化的过程。自20世纪60年代后期安德鲁斯JF(1968)和
Graef SP(1974)的初始动态数学消化模型以来,已经开发了更多和更复杂的模型来解
释显着的微生物相互作用和抑制\cite{yu13_mathem_model_anaer_diges_ad} 。这些模型包括具有更详细的抑制动力学和各种底物的
额外过程。然而,由于其涉及几组细菌的复杂动态过程,实现有效动力学常数的任务是困难
的。这里报告的研究目的是介绍水果和蔬菜废物的厌氧消化模拟,并从提出的数学模型中确
定一些未知参数\cite{donoso-bravo11_model_selec_ident_valid_anaer_diges} 。将P. Sosnowski等提出的简化模型应用于本研究,以获得模拟确
切条件的数学模型。

\section{举例-多哥共和国}
\label{sec:orge4d9602}
\subsection{经济表现}
\label{sec:orgb375be3}
多哥最近的经济表现相对强劲,过去五年国内生产总值平均增长5.5%。这种经济发展需要
稳定的能源供应;然而,该国的电气化率仍然很低,电力供应不可靠。 2014年,多哥46%的
人口(农村地区只有16%)获得电力,主要通过火力发电厂,尼日利亚和加纳的进口以及一
些水电站供电。虽然可再生能源在总能源消耗中所占的比例相对较高,为72.8%,但这包括
超过50%的传统固体生物质不可持续使用(世界银行,2018年)。

\subsection{潜力}
\label{sec:org1409827}
尽管如此,多哥(或其他西非国家)在分散可再生能源方面的潜力仍然很大。重点是从菠萝固体废物
中开发简单而有效的沼气生产技术,这将成为多哥常用能源的可持续替代品,如进口重质燃
料油或天然气。目标是找到导致最大沼气生产的条件,并开发一种具有低投资和运营成本的
可靠技术,这些技术可在发展中国家使用。根据 \cite{Arthur_2011} 等人的说法,2011年在加纳安装了大
约100个不同蒸煮器尺寸在\(10 m^3\)和\(30 m^3\)之间的工厂。在这些植物中,输入材
料由动物养殖的液体和固体残余物,人类排泄物,农业残余物和厨房垃圾组成。
\cite{Arthur_2011} 等人。
还指出,一般来说,只采用三种主要的消化池类型 - 固定式穹顶消化池,浮筒消化池和浦
新消化池 - 并且沼气主要用于不需要气体净化的烹饪用途。其原因在于生产电力,其中气
体纯度至关重要(\cite{G._Brener-A._Shinibekova-R._2017} ),沼气升级,沼气合成网络
(Čuček等,2017),或其他,在讨论的条件下,更先进的技术通常难以实施。至于输出侧
的生物浆料,通常将其用作肥料。更重要的是,\cite{Arthur_2011} 等人提到的100种植物中有一半。这
些作者已经访问过,他们指出只有一半(即25个)完全没有技术问题。 Amigun[ \cite{amigun08_commer_biofuel_indus_afric} ]等人详细描
述了非洲沼气生产设施故障的问题和原因。如下:

\begin{itemize}
\item 沼气池设计和施工不良,操作不当,缺乏维护。
\item 传播策略不佳以及发起人缺乏项目监督和后续行动。 •
\item 用户的所有权责任不佳。
\item 政府未能通过专注的能源政策支持沼气技术。
\end{itemize}

\subsection{沼气池}
\label{sec:org64aaa74}
关于在埃塞俄比亚传播沼气池的驱动因素和障碍的详细研究由Kamp和BermúdezForn(\cite{kamp16_emerg_domes_biogas_sector})
提供。在他们的论文中,他们指出,在过去二十年中,安装了2.5立方米到200立方米的沼气
池的1000个沼气生产厂中,只有大约40%仍在运营。在乌干达,Walekhwa[\cite{walekhwa09_biogas_energ_from_famil_sized_diges_ugand}]等人调查了与采用
沼气生产作为烹饪和照明的可持续燃料来源相关的社会经济,人口和政治主题。 (2009年)。
这项研究提到家庭大小的消化器,主要使用牛粪和其他来自农业的有机废物,显示出很大的
潜力,但并没有像它们那样广泛传播。这种沼气池似乎在印度,尼泊尔和中国的农村地区有
更强烈的应用,其中不同的沼气池类型以其本地产地或发明者命名。有关其操作的应用和优
化的文献可用 - 例如, [\cite{kalia88_devel_evaluat_fixed_dome_plug}
]比较了先进沼气厂类型和传统固定圆顶类型的沼气
产量,没有任何移动部件,也称为Janata模型。给出了整个系统的施工细节,其中包括1988
年的约500美元的材料投资。在进行所讨论的研究的地方,平均温度低于20°C,西非气候的
表现将非常有趣。 \cite{reddy04_lesson_from_pura_commun_biogas_projec} (2004)描述了印度农村社区中一个运行良好的小型沼气生产厂的
例子。整个社区为牛粪做贡献,作为奖励,人们以实惠的价格获得烹饪和照明的沼气,以及
由电动泵提供的饮用水。电力由柴油发动机产生,发电机由柴油和生产的沼气的混合物提供
燃料。 \cite{rajendran12_househ_biogas_diges_review} 提出了世界不同地区使用的沼气池类型的总体情况。针对
这些类型中的每一种给出了优点,缺点,构造细节和成本估计。描述了使用沼气的可能性,
并且再次强调,消化器的仔细操作对于成功生产沼气至关重要。马来西亚,作为世界第二大
棕榈油生产国,\cite{18_bioph_charac_palm_oil_mill} 等人。
(2017年)讨论了农村地区棕榈油厂废水产生的沼气,可以有效地用作替代能源。仅就水果
和蔬菜废物生产沼气而言,混合水果废物比混合水果蔬
菜废物提供10%的沼气产量。同一篇论文还讨论了一种可扩展的方法,使用Ca(OH)2将生
产的沼气中的甲烷含量提高到70%以上,以满足巴基斯坦的能源需求,同时可持续管理城市
固体废物的有机部分。从所有这些研究以及许多其他研究中可以清楚地看出,常用的技术足
以达到可接受的沼气产量;但是,在可靠性和维护方面,仍有很大的改进空间。这就是开发
一种简单,可靠和高效的低维护技术的重要原因。

\subsection{菠萝废物产生的沼气}
\label{sec:org25dc0c6}
多哥菠萝的季节性加工主要发生在分散的中小型企业。2016年,多哥的菠萝产
量为1,908吨。典型的企业加工1-2吨/天的新鲜菠萝。生产的菠萝产品
是干果和果汁,主要出口。得到的菠萝废物与鲜重相关约40%,即通常400-800kg / d可用
于生产沼气。可生物降解的有机废物包括果皮,核心,茎,冠废物和高含水量的废弃水果。
由于菠萝废物富含木质素,纤维素,半纤维素和其他碳水化合物,因此适合利用厌氧消化。
然而,这些化合物形成稳定的结壳,基本上通过水解使有机物质的生物分解变得复杂,因此
推荐在将废物送入蒸煮器之前通过压碎进行预处理。虽然生产的沼气主要用作可再生能源的
分散来源,但它也可以用作水果干燥过程的燃料。此外,上述废物产生的沼气有助于改善废
物管理,从而有助于减少发展中国家的环境污染。鉴于典型的非洲气候,液体消化物直接使
用,但就气候较少的发展中国家而言,通过各种蒸发方法增稠( \cite{Marek_Vondra_Vítězslav_Máša_Petr_Bobák} 等,2016)可能是有
益的。主要目标是通过水解,产酸,产乙酸和产甲烷作为尽可能有效,稳定和安全的菠萝废
物微生物转化为富含能量的沼气。讨论了影响该过程的参数,例如由
Baranowski(\cite{Cucek_L_Hjaila_K_Klemes_JJ_Kravanja_Z_2017} )。
为了探究它们如何影响沼气生产的性能,在本研究的第一阶段进行了实验室规模的实验。

\section{收获}
\label{sec:org23450e1}
这是我第一次尝试动手模仿专业论文的写作,以前虽然天天看wg21上的papers,但是却从未
思考过其排版,非常惭愧。通过这次练习,我熟练掌握了运用bibtex来管理文献,通过谷歌
学术来更为精确的定位文献。最重要的是激起了我向ISO:wg21提交proposal的勇气。

\section{致谢}
\label{sec:orgd42665c}
\begin{itemize}
\item \href{https://www.gnu.org/software/emacs/}{GNU Emacs} -- 如果没有Emacs这个世界上最强大的编辑软件的帮助下,我无法这么快速的完成论文任务
\item \href{https://scholar.google.com/}{Google Scholar} -- 最好用的学术搜索网站
\item \(\LaTeX\) -- linux下没有好用的排版软件,花了半天的功夫学习了下tex,因为对
mathjax/html/katex/emacs比较熟悉,所以还是比较容易上手的。
\item \href{https://github.com/jkitchin/org-ref}{org-ref} -- 比起纯手写\(LaTex\), 我更喜欢先在Emacs的org-mode中完成写作与排版,
剩下的tex格式转换交给xelatex或者pandoc等, 但是后者自带的文献
reference功能比较薄弱,而在org-ref这个elisp包的帮助下,使用bibtex来进行文献reference从未如此简
单!
\item \href{https://ctan.org/pkg/ctex?lang=en}{ctex} -- 由于tex的先天设计缺陷,其对cjk字体的支持非常不完善,所幸现在有ctex了。
\item \href{https://github.com/tumashu/pyim}{pyim} -- 一个Emacs中的拼音输入法,如果没有pyim, 我无法完成这篇中文论文。
\item \href{https://git-scm.com/}{git} -- 本文在书写中使用git来进行版本控制,虽然大部分push等操作都是通过emacs的
magit来完成的。好处是可以回滚至任一commit, 且能备份防灾。
\end{itemize}


\bibliographystyle{alpha}
\bibliography{manuscript}
\end{document}
