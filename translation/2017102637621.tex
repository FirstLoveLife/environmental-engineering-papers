%% ================================================================================
%% This LaTeX file was created by AbiWord.                                         
%% AbiWord is a free, Open Source word processor.                                  
%% More information about AbiWord is available at http://www.abisource.com/        
%% ================================================================================

\documentclass[a4paper,portrait,12pt]{article}
\usepackage[latin1]{inputenc}
\usepackage{calc}
\usepackage{setspace}
\usepackage{fixltx2e}
\usepackage{graphicx}
\usepackage{multicol}
\usepackage[normalem]{ulem}
%% Please revise the following command, if your babel
%% package does not support en-US
\usepackage[en]{babel}
\usepackage{color}
\usepackage{hyperref}
 
\begin{document}


\begin{flushleft}
Available online at www.sciencedirect.com
\end{flushleft}





\begin{flushleft}
CERAMICS
\end{flushleft}


\begin{flushleft}
INTERNATIONAL
\end{flushleft}





\begin{flushleft}
Ceramics International 39 (2013) 1385--1391
\end{flushleft}


\begin{flushleft}
www.elsevier.com/locate/ceramint
\end{flushleft}





\begin{flushleft}
Preparation and phosphorus recovery performance of porous
\end{flushleft}


\begin{flushleft}
calcium--silicate--hydrate
\end{flushleft}


\begin{flushleft}
Wei Guan, Fangying Jin, Qingkong Chen, Peng Yan, Qian Zhang
\end{flushleft}


\begin{flushleft}
Key Laboratory of Three Gorges Reservoir Region's Eco-Environment, Ministry of Education, Chongqing University, Chongqing 400045, PR China
\end{flushleft}


\begin{flushleft}
Received 18 June 2012; received in revised form 17 July 2012; accepted 24 July 2012
\end{flushleft}


\begin{flushleft}
Available online 1 August 2012
\end{flushleft}





\begin{flushleft}
Abstract
\end{flushleft}


\begin{flushleft}
Porous calcium--silicate--hydrate was synthesized and used to recover phosphorus from wastewater. The principal objective of this
\end{flushleft}


\begin{flushleft}
study was to explore the phosphorus recovery performance of porous calcium--silicate--hydrate prepared by different Ca/Si molar ratios.
\end{flushleft}


\begin{flushleft}
Phosphorus recovery mechanism was also investigated via Field Emission Scanning Electron Microscopy (FESEM), Energy Dispersive
\end{flushleft}


\begin{flushleft}
Spectrum (EDS), Brunauer--Emmett--Teller (BET) and X-ray Diffraction (XRD). The law of Ca2 þ release was the key of phosphorus
\end{flushleft}


\begin{flushleft}
recovery performance. Different Ca/Si molar ratios resulted in the changes of pore structures. The increase of specific surface area and
\end{flushleft}


\begin{flushleft}
the increase in concentration of Ca2 þ release were well agreement together. The Ca/Si molar ratio of 1.6 for porous calcium--silicate--
\end{flushleft}


\begin{flushleft}
hydrate is more proper to recover phosphorus. The pore structure of porous calcium--silicate--hydrate provided a local condition to
\end{flushleft}


\begin{flushleft}
maintain a high concentration of Ca2 þ release. Porous calcium--silicate--hydrate could release a proper concentration of Ca2 þ and OH ?
\end{flushleft}


\begin{flushleft}
to maintain the pH values at 8.5--9.5. This condition was beneficial to the formation of hydroxyapatite. Phosphorus content of porous
\end{flushleft}


\begin{flushleft}
calcium--silicate--hydrate reached 18.64\% after phosphorus recovery.
\end{flushleft}


\begin{flushleft}
\& 2012 Elsevier Ltd and Techna Group S.r.l. All rights reserved.
\end{flushleft}


\begin{flushleft}
Keywords: Calcium--silicate--hydrate; Phosphorus recovery; Porous structure; Preparation
\end{flushleft}





\begin{flushleft}
1. Introduction
\end{flushleft}


\begin{flushleft}
Phosphorus not only plays an important role in water
\end{flushleft}


\begin{flushleft}
eutrophication, but also is a non-renewable and irreplaceable
\end{flushleft}


\begin{flushleft}
resource [1]. Global phosphate rock resources will be
\end{flushleft}


\begin{flushleft}
completely exhausted in 100 years. Therefore recovery from
\end{flushleft}


\begin{flushleft}
wastewater has been considered as the only way for developing sustainable phosphorus resource [2,3]. Phosphorus
\end{flushleft}


\begin{flushleft}
recovery from wastewater in the form of hydroxyapatite is a
\end{flushleft}


\begin{flushleft}
common and simple method [4--6]. However, in the formation process of hydroxyapatite, super-saturation is a common phenomenon. The optimal pH value for the formation
\end{flushleft}


\begin{flushleft}
of hydroxyapatite is in the range of 10.5--12.5 [7]. This pH
\end{flushleft}


\begin{flushleft}
value is too high to biochemical treatment system where the
\end{flushleft}


\begin{flushleft}
pH value is located between 6.0 and 9.0 [8]. For the
\end{flushleft}


\begin{flushleft}
phosphorus removal by biological treatment process assisted
\end{flushleft}


\begin{flushleft}
with chemical method, high-pH condition made the two
\end{flushleft}


\begin{flushleft}
processes difficult to coordinate. Higher pH value also
\end{flushleft}


\begin{flushleft}
n
\end{flushleft}





\begin{flushleft}
Corresponding author. Tel./fax: þ86 023 65127537.
\end{flushleft}


\begin{flushleft}
E-mail address: jfy@cqu.edu.cn (F. Ji).
\end{flushleft}





\begin{flushleft}
increased significant competition between carbonate and
\end{flushleft}


\begin{flushleft}
calcium [9]. Meanwhile, the cost of chemical treatment
\end{flushleft}


\begin{flushleft}
increased and the effective phosphorus composition of the
\end{flushleft}


\begin{flushleft}
final products are decreased [10].
\end{flushleft}


\begin{flushleft}
Calcium--silicate--hydrate (CSH) as seed crystal could be
\end{flushleft}


\begin{flushleft}
used to remove phosphorus from wastewater through
\end{flushleft}


\begin{flushleft}
crystallization of hydroxyapatite [11]. Ca2 þ and OH ?
\end{flushleft}


\begin{flushleft}
were released from CSH and reacted with the phosphate
\end{flushleft}


\begin{flushleft}
to form hydroxyapatite under the condition of pH ¼ 8.5--
\end{flushleft}


\begin{flushleft}
9.5. However, phosphorus content of CSH was too low to
\end{flushleft}


\begin{flushleft}
recover phosphorus in practical application [12,13]. The
\end{flushleft}


\begin{flushleft}
efficiency of Ca2 þ and OH ? release was related to the
\end{flushleft}


\begin{flushleft}
pore structure of CSH [14] and affected phosphorus
\end{flushleft}


\begin{flushleft}
recovery performance [15,16]. Ca/Si molar ratio has a
\end{flushleft}


\begin{flushleft}
significant effect on the pore structure of CSH [17--19]. In
\end{flushleft}


\begin{flushleft}
the part of phosphorus recovery performance, systematic
\end{flushleft}


\begin{flushleft}
relationship between Ca/Si molar ratio and phosphorus
\end{flushleft}


\begin{flushleft}
recovery performance has not been established. The
\end{flushleft}


\begin{flushleft}
mechanism of the influence of Ca/Si molar ratio on
\end{flushleft}


\begin{flushleft}
phosphorus recovery performance and the law of Ca2 þ
\end{flushleft}


\begin{flushleft}
and OH ? release were still unclear. So it was challenge to
\end{flushleft}





\begin{flushleft}
0272-8842/\$ - see front matter \& 2012 Elsevier Ltd and Techna Group S.r.l. All rights reserved.
\end{flushleft}


\begin{flushleft}
http://dx.doi.org/10.1016/j.ceramint.2012.07.079
\end{flushleft}





\begin{flushleft}
\newpage
W. Guan et al. / Ceramics International 39 (2013) 1385--1391
\end{flushleft}





1386





\begin{flushleft}
determine a proper Ca/Si molar ratio for CSH to recover
\end{flushleft}


\begin{flushleft}
phosphorus.
\end{flushleft}


\begin{flushleft}
The main aim of the research is to find a proper Ca/Si
\end{flushleft}


\begin{flushleft}
molar ratio for CSH to recover phosphorus. The originality and importance of this paper are highlighted by the
\end{flushleft}


\begin{flushleft}
following three points:
\end{flushleft}


\begin{flushleft}
(1) Porous calcium--silicate--hydrate was synthesized by
\end{flushleft}


\begin{flushleft}
carbide residue and white carbon black with a dynamic
\end{flushleft}


\begin{flushleft}
hydrothermal method [20,21]. The influence of Ca/Si
\end{flushleft}


\begin{flushleft}
molar ratio on phosphorus recovery performance was
\end{flushleft}


\begin{flushleft}
investigated.
\end{flushleft}


\begin{flushleft}
(2) The relationship between pore structure and the law of
\end{flushleft}


\begin{flushleft}
Ca2 þ and OH ? release was established by the Avrami
\end{flushleft}


\begin{flushleft}
kinetic model.
\end{flushleft}


\begin{flushleft}
(3) Phosphorus recovery mechanism was studied by
\end{flushleft}


\begin{flushleft}
FESEM, EDS, BET and XRD on the basis of an indepth critical investigation.
\end{flushleft}





\begin{flushleft}
2. Materials and methods
\end{flushleft}


\begin{flushleft}
2.1. Preparation of porous calcium--silicate--hydrate
\end{flushleft}


\begin{flushleft}
Porous calcium--silicate--hydrate was synthesized with
\end{flushleft}


\begin{flushleft}
carbide residue (providing Ca) and white carbon black
\end{flushleft}


\begin{flushleft}
(providing Si). Carbide residue (calcareous, hoar and
\end{flushleft}


\begin{flushleft}
powdery) was obtained from Chongqing Changshou Chemical Co. Ltd. and calcined at 700 1C for 2 h. White
\end{flushleft}


\begin{flushleft}
carbon black (Particles present spherical with homogeneous diameter) was purchased from Chongqing Jianfeng
\end{flushleft}


\begin{flushleft}
chemical Co. Ltd. Chemical constituents of carbide residue
\end{flushleft}


\begin{flushleft}
and white carbon black are shown in Table 1. Tobermorite
\end{flushleft}


\begin{flushleft}
(formula Ca5Si6O16(OH)2 ? 7H2O; Theoretical Ca/Si molar
\end{flushleft}


\begin{flushleft}
ratio of 0.83), a kind of calcium--silicate--hydrate, was
\end{flushleft}


\begin{flushleft}
purchased from Hdlapp and Ratus (Shanghai) Co. Ltd.
\end{flushleft}


\begin{flushleft}
The phosphorus solution was adjusted by adding KH2PO4
\end{flushleft}


\begin{flushleft}
(Analytical reagent, Chongqing Boyi chemical reagent Co.
\end{flushleft}


\begin{flushleft}
Ltd.) to prepare solution with initial phosphorus concentration of 100 mg/L. The above materials and chemicals
\end{flushleft}


\begin{flushleft}
were placed into sealed bottles for storage.
\end{flushleft}


\begin{flushleft}
Carbide residue and white carbon black were mixed, and
\end{flushleft}


\begin{flushleft}
the Ca/Si molar ratios were controlled at 0.6, 1.0, 1.6 and
\end{flushleft}


\begin{flushleft}
2.2. The mixture was then added to prepared slurries. The
\end{flushleft}


\begin{flushleft}
slurry was hydrothermally reacted at 170 1C for 6 h and
\end{flushleft}


\begin{flushleft}
taken out when the temperature was reduced to the natural
\end{flushleft}





\begin{flushleft}
condition. The hydrothermal reaction was carried out with
\end{flushleft}


\begin{flushleft}
a liquid/solid ratio of 30. The obtained products were dried
\end{flushleft}


\begin{flushleft}
at 105 1C for 2 h, and then were ground through a sieve of
\end{flushleft}


\begin{flushleft}
200 meshes. The prepared samples with Ca/Si molar ratios
\end{flushleft}


\begin{flushleft}
of 0.6, 1.0, 1.6 and 2.2 were denoted as CSH: Ca/Si ¼ 0.6,
\end{flushleft}


\begin{flushleft}
CSH: Ca/Si ¼ 1.0, CSH: Ca/Si¼ 1.6 and CSH: Ca/Si ¼ 2.2,
\end{flushleft}


\begin{flushleft}
respectively.
\end{flushleft}





\begin{flushleft}
2.2. Evaluation of phosphorus recovery performance
\end{flushleft}


\begin{flushleft}
Synthetic solution (1 L) was added to bottles, respectively. Certain quality of samples (500, 1000, 2000, 3000,
\end{flushleft}


\begin{flushleft}
4000, 5000 and 6000 mg) were added to these bottles
\end{flushleft}


\begin{flushleft}
respectively and shaken at 40 r/min under controlled temperature conditions (20 1C). Phosphorus concentration of
\end{flushleft}


\begin{flushleft}
supernatant was measured according to the molybdenum--
\end{flushleft}


\begin{flushleft}
blue ascorbic acid method (The relative error of data is
\end{flushleft}


\begin{flushleft}
0.3\%) with a Unico spectrophotometer (UV-2102PCS,
\end{flushleft}


\begin{flushleft}
Shanghai Unico Instruments Co., Ltd., China) [22]. The
\end{flushleft}


\begin{flushleft}
solid samples after reaction were then separated from the
\end{flushleft}


\begin{flushleft}
removed synthetic solution, and were added again to
\end{flushleft}


\begin{flushleft}
synthetic solution with initial phosphorus concentration of
\end{flushleft}


\begin{flushleft}
100 mg/L. This experiment was repeated for several times
\end{flushleft}


\begin{flushleft}
until the phosphorous concentration was kept unchanged
\end{flushleft}


\begin{flushleft}
with the addition of samples. Finally, the produced sediments were separated from removed synthetic solution,
\end{flushleft}


\begin{flushleft}
dried and weighted. Phosphorus contents of samples after
\end{flushleft}


\begin{flushleft}
phosphorus recovery (P) were calculated by Eq. (1), where
\end{flushleft}


\begin{flushleft}
Ct is the restrained phosphorus concentration in synthetic
\end{flushleft}


\begin{flushleft}
solution (mg/L), v is the volume of the solution (L), w is the
\end{flushleft}


\begin{flushleft}
mass of produced sediment after phosphorus recovery (mg)
\end{flushleft}


\begin{flushleft}
and C0 is the initial phosphorus concentration (mg/L).
\end{flushleft}


\begin{flushleft}
P¼
\end{flushleft}





\begin{flushleft}
ðc0 ?ct Þv
\end{flushleft}


100\%


\begin{flushleft}
w
\end{flushleft}





\begin{flushleft}
ð1Þ
\end{flushleft}





\begin{flushleft}
4 g of samples (CSH: Ca/Si ¼ 0.6, CSH: Ca/Si ¼ 1.0,
\end{flushleft}


\begin{flushleft}
CSH: Ca/Si¼ 1.6, CSH: Ca/Si¼ 2.2 and tobermorite) were
\end{flushleft}


\begin{flushleft}
immersed in 1 L of deionised water respectively contained
\end{flushleft}


\begin{flushleft}
in a glass bottle, generating samples with a solution
\end{flushleft}


\begin{flushleft}
concentration of 4 g/L. The bottle was placed on an
\end{flushleft}


\begin{flushleft}
agitation table and shaken at 40 r/min under controlled
\end{flushleft}


\begin{flushleft}
temperature conditions (20 1C). Samples of solution were
\end{flushleft}


\begin{flushleft}
taken after 5, 10, 15, 20, 40, 60, 80 and 100 min of
\end{flushleft}


\begin{flushleft}
agitation. Ca2 þ concentration of the samples was determined by EDTA titration (The relative error of data is
\end{flushleft}


0.05\%) [23].





\begin{flushleft}
Table 1
\end{flushleft}


\begin{flushleft}
Chemical components of carbide residue and white carbon black.
\end{flushleft}


\begin{flushleft}
Chemical Components (Contents)/\%
\end{flushleft}





\begin{flushleft}
carbide residue
\end{flushleft}


\begin{flushleft}
white carbon black
\end{flushleft}





\begin{flushleft}
CaO
\end{flushleft}





\begin{flushleft}
SiO2
\end{flushleft}





\begin{flushleft}
Al2O3
\end{flushleft}





\begin{flushleft}
SO2
\end{flushleft}





\begin{flushleft}
MgO
\end{flushleft}





\begin{flushleft}
Fe2O3
\end{flushleft}





\begin{flushleft}
SrO
\end{flushleft}





\begin{flushleft}
NaOH
\end{flushleft}





\begin{flushleft}
CuO
\end{flushleft}





\begin{flushleft}
H2O
\end{flushleft}





79.34


0.08





3.57


97.46





2.14


0.16





1.22


1.82





0.62


--





0.21


0.03





0.26


--





--


0.29





--


0.02





12.64


0.14





\begin{flushleft}
\newpage
W. Guan et al. / Ceramics International 39 (2013) 1385--1391
\end{flushleft}





1387





\begin{flushleft}
2.3. Characterization methods
\end{flushleft}


\begin{flushleft}
XRD patterns were collected in a XD-2 instrument
\end{flushleft}


\begin{flushleft}
(Persee, China) using Cu Ka radiation. FESEM images
\end{flushleft}


\begin{flushleft}
were collected on an S-4800 field emission scanning
\end{flushleft}


\begin{flushleft}
electron microscope (Hitachi, Japan). BET surface areas
\end{flushleft}


\begin{flushleft}
were measured by nitrogen adsorption at 77.35 K on an
\end{flushleft}


\begin{flushleft}
ASAP-2010 adsorption apparatus (Micromeritics, USA).
\end{flushleft}


\begin{flushleft}
3. Results and discussion
\end{flushleft}


\begin{flushleft}
3.1. Phosphorus recovery performance of porous calcium--
\end{flushleft}


\begin{flushleft}
silicate--hydrate
\end{flushleft}





\begin{flushleft}
Restrained phosphorus concentration/(mg/L)
\end{flushleft}





100





\begin{flushleft}
CSH:Ca/Si=0.6
\end{flushleft}


\begin{flushleft}
CSH:Ca/Si=1.0
\end{flushleft}


\begin{flushleft}
CSH:Ca/Si=1.6
\end{flushleft}


\begin{flushleft}
CSH:Ca/Si=2.2
\end{flushleft}


\begin{flushleft}
Tobermorite
\end{flushleft}





80





60





40





20





0





\begin{flushleft}
Restrained phosphorus concentration/(mg/L)
\end{flushleft}





110


\begin{flushleft}
CSH:Ca/Si=0.6
\end{flushleft}


\begin{flushleft}
CSH:Ca/Si=1.0
\end{flushleft}


\begin{flushleft}
CSH:Ca/Si=1.6
\end{flushleft}


\begin{flushleft}
CSH:Ca/Si=2.2
\end{flushleft}


\begin{flushleft}
Tobermorite
\end{flushleft}





100


90


80


70


60


50


40


30


20


10


0


0





20





40





60





80





100





\begin{flushleft}
Reaction time/min
\end{flushleft}


\begin{flushleft}
Fig. 1. Influence of reaction time on restrained phosphorus concentration.
\end{flushleft}





0





1000





2000





3000





4000





5000





6000





\begin{flushleft}
Dosage of samples/(mg/L)
\end{flushleft}





\begin{flushleft}
Fig. 2. Influence of samples dosage on restrained phosphorus concentration.
\end{flushleft}





\begin{flushleft}
Restrained phosphorus concentration/(mg/L)
\end{flushleft}





\begin{flushleft}
The influence of reaction time on restrained phosphorus
\end{flushleft}


\begin{flushleft}
concentration is shown in Fig. 1. A sharp decrease in
\end{flushleft}


\begin{flushleft}
phosphorus concentration was observed during the first
\end{flushleft}


\begin{flushleft}
20 min. phosphorus concentration declined slightly with
\end{flushleft}


\begin{flushleft}
prolonged time. The difference of the restrained phosphorus concentration was significant when the reaction
\end{flushleft}


\begin{flushleft}
reached equilibrium at 60 min. Restrained phosphorus
\end{flushleft}


\begin{flushleft}
concentration reached 22.19 mg/L when the molar ratio
\end{flushleft}


\begin{flushleft}
of Ca/Si was 0.6. With the increase in Ca/Si molar ratio,
\end{flushleft}


\begin{flushleft}
phosphorus removal capacity of samples is improved
\end{flushleft}


\begin{flushleft}
significantly. Restrained phosphorus concentration was
\end{flushleft}


\begin{flushleft}
2.16 mg/L when the molar ratio of Ca/Si was 2.2.
\end{flushleft}


\begin{flushleft}
Fig. 2 shows the removal of phosphorus with different
\end{flushleft}


\begin{flushleft}
samples dosing. Efficiency of phosphorus removal was
\end{flushleft}


\begin{flushleft}
improved when the dosage increased and the highest
\end{flushleft}


\begin{flushleft}
removal efficiency was obtained at 4000 mg/L. Then, the
\end{flushleft}


\begin{flushleft}
phosphorus removal efficiency almost remained stable with
\end{flushleft}


\begin{flushleft}
the further increase in samples dosing. Comparatively
\end{flushleft}


\begin{flushleft}
speaking, CSH: Ca/Si ¼ 2.2 showed the highest phosphorus
\end{flushleft}


\begin{flushleft}
removal efficiency. Restrained phosphorus concentration
\end{flushleft}


\begin{flushleft}
was only 2.16 mg/L and the mass of sediment was
\end{flushleft}


\begin{flushleft}
3750 mg. However, the phosphorus content of CSH:
\end{flushleft}


\begin{flushleft}
Ca/Si ¼ 2.2 was only 2.6\%. Phosphorus content of samples
\end{flushleft}


\begin{flushleft}
could be enhanced due to circulation of phosphorus
\end{flushleft}


\begin{flushleft}
removal.
\end{flushleft}





100





80





60





40





\begin{flushleft}
CSH:Ca/Si=0.6
\end{flushleft}


\begin{flushleft}
CSH:Ca/Si=1.0
\end{flushleft}


\begin{flushleft}
CSH:Ca/Si=1.6
\end{flushleft}


\begin{flushleft}
CSH:Ca/Si=2.2
\end{flushleft}


\begin{flushleft}
Tobermorite
\end{flushleft}





20





0


0





4


8


12


\begin{flushleft}
Number of times of phosphorus removal by samples
\end{flushleft}





16





\begin{flushleft}
Fig. 3. Changes of restrained phosphorus concentration by circulation of
\end{flushleft}


\begin{flushleft}
phosphorus removal.
\end{flushleft}





\begin{flushleft}
Samples were separated from the synthetic solution
\end{flushleft}


\begin{flushleft}
removed, and then added to synthetic solution with
\end{flushleft}


\begin{flushleft}
initial phosphorus concentration 100 mg/L. Changes of
\end{flushleft}


\begin{flushleft}
restrained phosphorus concentration are shown in Fig. 3.
\end{flushleft}


\begin{flushleft}
Phosphorus removal performance of CSH: Ca/Si ¼ 2.2
\end{flushleft}


\begin{flushleft}
maintained well during first 3 times, and ceased after the
\end{flushleft}


\begin{flushleft}
12th time. Phosphorus content of CSH: Ca/Si ¼ 2.2 was
\end{flushleft}


\begin{flushleft}
14.10\%, while phosphorus content of CSH: Ca/Si ¼ 1.6
\end{flushleft}


\begin{flushleft}
reached 18.64\%. CSH: Ca/Si¼ 1.6 had a higher phosphorus recovery performance compared with CSH: Ca/
\end{flushleft}


\begin{flushleft}
Si¼ 2.2. Phosphorus removal performance of samples was
\end{flushleft}


\begin{flushleft}
related to pH values. With the times of phosphorus
\end{flushleft}


\begin{flushleft}
removal prolonged, pH values decreased (Fig. 4). As seen,
\end{flushleft}


\begin{flushleft}
CSH: Ca/Si¼ 2.2 contributed to a range of high pH values
\end{flushleft}


\begin{flushleft}
(pH ¼ 9.8--10.2) at first 3 times, and declined sharply at the
\end{flushleft}


\begin{flushleft}
4th time (pH ¼ 8.5). CSH: Ca/Si ¼ 1.6 could maintain high
\end{flushleft}


\begin{flushleft}
pH values (pH ¼ 8.5--9.5) for a long time (10 times of
\end{flushleft}





\begin{flushleft}
\newpage
W. Guan et al. / Ceramics International 39 (2013) 1385--1391
\end{flushleft}





1388





\begin{flushleft}
CSH:Ca/Si=0.6
\end{flushleft}


\begin{flushleft}
CSH:Ca/Si=1.0
\end{flushleft}


\begin{flushleft}
CSH:Ca/Si=1.6
\end{flushleft}


\begin{flushleft}
CSH:Ca/Si=2.2
\end{flushleft}


\begin{flushleft}
Tobermorite
\end{flushleft}





\begin{flushleft}
pH values
\end{flushleft}





10





9





8





7





0





2


4


6


8


10


12


14


\begin{flushleft}
Number of times of phosphorus removal by samples
\end{flushleft}





16





\begin{flushleft}
ratio resulted in smaller pore size, larger specific surface
\end{flushleft}


\begin{flushleft}
area and pore volume.
\end{flushleft}


\begin{flushleft}
The surface structure of tobermorite, CSH: Ca/Si ¼ 1.6
\end{flushleft}


\begin{flushleft}
and CSH: Ca/Si¼ 2.2 was examined by FESEM observations and EDS analyses (Fig. 6). Compared with tobermorite, CSH: Ca/Si¼ 1.6 had an obverse fibrous-network
\end{flushleft}


\begin{flushleft}
structure with a large number of mesopores. CSH:
\end{flushleft}


\begin{flushleft}
Ca/Si ¼ 2.2 had massive flake crystal besides fibrous-network structure. EDS analyses confirmed that the coarse
\end{flushleft}


\begin{flushleft}
surface of tobermorite, CSH: Ca/Si ¼ 1.6 and CSH: Ca/
\end{flushleft}


\begin{flushleft}
Si¼ 2.2 consisted predominantly of Ca and Si. Ca/Si molar
\end{flushleft}


\begin{flushleft}
ratio was 0.8, 1.5 and 2.0, respectively. Ca/Si molar ratio
\end{flushleft}


\begin{flushleft}
of materials decreased after synthesis due to the loss of
\end{flushleft}


\begin{flushleft}
partial Ca2 þ when filtering slurry. Thus, the single phosphorus removal efficiency of CSH increased with the
\end{flushleft}


\begin{flushleft}
increased specific surface area.
\end{flushleft}





\begin{flushleft}
Fig. 4. Changes of pH values by circulation of phosphorus removal.
\end{flushleft}





\begin{flushleft}
3.3. Kinetics of Ca2 þ release
\end{flushleft}


\begin{flushleft}
The experiments showed that Ca2 þ concentrations
\end{flushleft}


\begin{flushleft}
increased with the increase of Ca/Si molar ratio (Fig. 7).
\end{flushleft}


\begin{flushleft}
Concentrations of Ca2 þ released from tobermorite, CSH:
\end{flushleft}


\begin{flushleft}
Ca/Si¼ 1.6 and CSH: Ca/Si¼ 2.2 were 2.10, 3.56, 4.91 mg/g,
\end{flushleft}


\begin{flushleft}
respectively. The experimental capacities of Ca2 þ release
\end{flushleft}


\begin{flushleft}
were plotted according to Avrami kinetic model equation
\end{flushleft}


\begin{flushleft}
(Eq. (2)) [25].
\end{flushleft}





300





\begin{flushleft}
Volume/(cm3/g)
\end{flushleft}





250


200





\begin{flushleft}
CSH:Ca/Si=0.6
\end{flushleft}


\begin{flushleft}
CSH:Ca/Si=1.0
\end{flushleft}


\begin{flushleft}
CSH:Ca/Si=1.6
\end{flushleft}


\begin{flushleft}
CSH:Ca/Si=2.2
\end{flushleft}


\begin{flushleft}
Tobermorite
\end{flushleft}





150





\begin{flushleft}
?lnð1?xÞ ¼ ktn
\end{flushleft}





100


50


0


0.0





0.2





0.4


0.6


\begin{flushleft}
Relative pressure p/p0
\end{flushleft}





0.8





1.0





\begin{flushleft}
Fig. 5. Nitrogen adsorption--desorption isotherms on samples.
\end{flushleft}





\begin{flushleft}
phosphorus removal). This condition was beneficial to
\end{flushleft}


\begin{flushleft}
circulation of phosphorus removal.
\end{flushleft}





\begin{flushleft}
ð2Þ
\end{flushleft}





\begin{flushleft}
where k is the kinetic constant, n is the characteristic
\end{flushleft}


\begin{flushleft}
constant of solid, t is the reaction time (min) and x (x ¼ Ct/
\end{flushleft}


\begin{flushleft}
Cmax, Ct is concentration of time t (mg/L) and Cmax is
\end{flushleft}


\begin{flushleft}
concentration of the maximum (mg/L)) is the fraction
\end{flushleft}


\begin{flushleft}
conversion. The characteristic constant n was 0.9019.
\end{flushleft}


\begin{flushleft}
The kinetic constants were determined by fitting the
\end{flushleft}


\begin{flushleft}
Avrami kinetic model to the experimental data obtained
\end{flushleft}


\begin{flushleft}
from Fig. 6 (Table 2). The high correlation coefficients
\end{flushleft}


\begin{flushleft}
(R2 4 0.99) indicated that this model could describe the
\end{flushleft}


\begin{flushleft}
law of Ca2 þ release well.
\end{flushleft}


\begin{flushleft}
As shown in Table 2, k became larger with increasing
\end{flushleft}


\begin{flushleft}
Ca/Si molar ratio. Combined with specific surface area (S)
\end{flushleft}


\begin{flushleft}
of materials, a relationship between k and S could be
\end{flushleft}


\begin{flushleft}
established (Eq. (3)).
\end{flushleft}





\begin{flushleft}
3.2. The pore structure of porous calcium--silicate--hydrate
\end{flushleft}





\begin{flushleft}
k ¼ 0:022S0:292
\end{flushleft}





\begin{flushleft}
Nitrogen adsorption--desorption isotherms on samples
\end{flushleft}


\begin{flushleft}
are shown in Fig. 5. The results suggested the phenomenon
\end{flushleft}


\begin{flushleft}
of adsorption hysteresis loop. That means mesopore or
\end{flushleft}


\begin{flushleft}
narrow gap pore existed on samples [24]. Adsorption in
\end{flushleft}


\begin{flushleft}
mespore occurred mainly in medium pressure region
\end{flushleft}


\begin{flushleft}
(0:4o p=p0 o 0:9). With the increase in Ca/Si molar ratio,
\end{flushleft}


\begin{flushleft}
the phenomenon of adsorption hysteresis loop became
\end{flushleft}


\begin{flushleft}
obvious and the adsorption curve increased. Specific surface areas of CSH: Ca/Si ¼ 0.6, CSH: Ca/Si ¼ 1.0, CSH:
\end{flushleft}


\begin{flushleft}
Ca/Si ¼ 1.6, CSH: Ca/Si¼ 2.2 and tobermorite were 11.91,
\end{flushleft}


\begin{flushleft}
59.67, 113.36, 121.03 and 49.85 m2/g, respectively. Pore
\end{flushleft}


\begin{flushleft}
volumes of these samples were 0.07, 0.30, 0.52, 0.65 and
\end{flushleft}


\begin{flushleft}
0.15 cm3/g, correspondingly. The increase of Ca/Si molar
\end{flushleft}





\begin{flushleft}
According to Eq. (3), specific surface area of samples
\end{flushleft}


\begin{flushleft}
and the rate of Ca2 þ release were in good agreement with
\end{flushleft}


\begin{flushleft}
each other. The relationship between specific surface area
\end{flushleft}


\begin{flushleft}
and dissolved concentration of Ca2 þ was obtained by
\end{flushleft}


\begin{flushleft}
substituting Eq. (3) into Eq. (2).
\end{flushleft}





\begin{flushleft}
R2 ¼ 0:9135
\end{flushleft}





\begin{flushleft}
ð3Þ
\end{flushleft}





\begin{flushleft}
?lnð1?xÞ ¼ 0:022S 0:292 t0:9019
\end{flushleft}





\begin{flushleft}
ð4Þ
\end{flushleft}


\begin{flushleft}
2þ
\end{flushleft}





\begin{flushleft}
According to Eq. (4), concentration of Ca release was
\end{flushleft}


\begin{flushleft}
related to specific surface area. This result demonstrated
\end{flushleft}


\begin{flushleft}
the influence of Ca/Si molar ratio on phosphorus recovery
\end{flushleft}


\begin{flushleft}
capacity. Ca/Si molar ratio affected the pore structure and
\end{flushleft}


\begin{flushleft}
the capacity of Ca2 þ release. Ca2 þ was released faster due
\end{flushleft}


\begin{flushleft}
to larger specific surface area. Porous structure provided a
\end{flushleft}





\begin{flushleft}
\newpage
W. Guan et al. / Ceramics International 39 (2013) 1385--1391
\end{flushleft}





1389





\begin{flushleft}
Fig. 6. FESEM observations and EDS analyses. (a) surface of tobermorite; (b) chemical analysis of tobermorite; (c) surface of CSH: Ca/Si¼ 1.6;
\end{flushleft}


\begin{flushleft}
(d) chemical analysis of CSH: Ca/Si¼1.6; (e) surface of CSH: Ca/Si¼2.2; and (f): chemical analysis of CSH: Ca/Si¼2.2.
\end{flushleft}





\begin{flushleft}
CSH:Ca/Si=0.6
\end{flushleft}


\begin{flushleft}
CSH:Ca/Si=1.6
\end{flushleft}


\begin{flushleft}
Tobermorite
\end{flushleft}





\begin{flushleft}
Ca2+ release/(mg/g)
\end{flushleft}





6





\begin{flushleft}
Table 2
\end{flushleft}


\begin{flushleft}
Correlation equations and rate constants for the Avrami kinetic model
\end{flushleft}


\begin{flushleft}
describing Ca2 þ release.
\end{flushleft}





\begin{flushleft}
CSH:Ca/Si=1.0
\end{flushleft}


\begin{flushleft}
CSH:Ca/Si=2.2
\end{flushleft}





4





2





0


0





20





40


60


\begin{flushleft}
Reaction time/min
\end{flushleft}





80





\begin{flushleft}
Fig. 7. Concentrations of Ca2 þ released from samples.
\end{flushleft}





100





\begin{flushleft}
Samples
\end{flushleft}





\begin{flushleft}
Correlation equation
\end{flushleft}





\begin{flushleft}
k
\end{flushleft}





\begin{flushleft}
R2
\end{flushleft}





\begin{flushleft}
CSH: Ca/Si¼ 0.6
\end{flushleft}


\begin{flushleft}
Tobermorite
\end{flushleft}


\begin{flushleft}
CSH: Ca/Si¼ 1.0
\end{flushleft}


\begin{flushleft}
CSH: Ca/Si¼ 1.6
\end{flushleft}


\begin{flushleft}
CSH: Ca/Si¼ 2.2
\end{flushleft}





\begin{flushleft}
?lnð1?xÞ ¼ 0:0472t0:9019
\end{flushleft}


\begin{flushleft}
?lnð1?xÞ ¼ 0:0604t0:9019
\end{flushleft}


\begin{flushleft}
?lnð1?xÞ ¼ 0:0741t0:9019
\end{flushleft}


\begin{flushleft}
?lnð1?xÞ ¼ 0:0866t0:9019
\end{flushleft}


\begin{flushleft}
?lnð1?xÞ ¼ 0:0966t0:9019
\end{flushleft}





0.0472


0.0604


0.0741


0.0866


0.0966





0.9955


0.9943


0.9933


0.9980


0.9991





\begin{flushleft}
local condition to maintain a high concentration of Ca2 þ
\end{flushleft}


\begin{flushleft}
release. Compared CSH: Ca/Si ¼ 1.6 with CSH: Ca/Si ¼
\end{flushleft}


\begin{flushleft}
2.2, the former had higher phosphorus recovery performance in fact. So the law of Ca2 þ release is the key of
\end{flushleft}


\begin{flushleft}
phosphorus recovery performance. CSH: Ca/Si ¼ 1.6 could
\end{flushleft}


\begin{flushleft}
release a suitable concentration of Ca2 þ and OH ? to
\end{flushleft}





\begin{flushleft}
\newpage
W. Guan et al. / Ceramics International 39 (2013) 1385--1391
\end{flushleft}





1390





1200





\begin{flushleft}
Intensity/(a.u.)
\end{flushleft}





900





\begin{flushleft}
A
\end{flushleft}


\begin{flushleft}
A
\end{flushleft}





600





\begin{flushleft}
A:
\end{flushleft}


\begin{flushleft}
B:
\end{flushleft}


\begin{flushleft}
C:
\end{flushleft}


\begin{flushleft}
D:
\end{flushleft}





\begin{flushleft}
C
\end{flushleft}





\begin{flushleft}
A
\end{flushleft}


\begin{flushleft}
A
\end{flushleft}





\begin{flushleft}
AA
\end{flushleft}





\begin{flushleft}
A
\end{flushleft}


\begin{flushleft}
C
\end{flushleft}





\begin{flushleft}
CSH:Ca/Si=2.2
\end{flushleft}





\begin{flushleft}
A
\end{flushleft}


\begin{flushleft}
A
\end{flushleft}


\begin{flushleft}
A
\end{flushleft}





\begin{flushleft}
A
\end{flushleft}





\begin{flushleft}
A
\end{flushleft}





\begin{flushleft}
B
\end{flushleft}





\begin{flushleft}
Jennite
\end{flushleft}


\begin{flushleft}
Xonotlite
\end{flushleft}


\begin{flushleft}
Ca(OH)2
\end{flushleft}


\begin{flushleft}
SiO2
\end{flushleft}





\begin{flushleft}
CSH:Ca/Si=1.6
\end{flushleft}





\begin{flushleft}
B
\end{flushleft}





\begin{flushleft}
B
\end{flushleft}





\begin{flushleft}
B
\end{flushleft}





\begin{flushleft}
B
\end{flushleft}





\begin{flushleft}
CSH:Ca/Si=1.0
\end{flushleft}





300


\begin{flushleft}
B B
\end{flushleft}


\begin{flushleft}
DD
\end{flushleft}





\begin{flushleft}
B
\end{flushleft}





\begin{flushleft}
B
\end{flushleft}





\begin{flushleft}
B
\end{flushleft}





\begin{flushleft}
CSH:Ca/Si=0.6
\end{flushleft}





0


10





20





30





40


\begin{flushleft}
2$\theta$/ (°)
\end{flushleft}





50





60





70





\begin{flushleft}
Fig. 8. X-ray diffraction (XRD) patterns of samples.
\end{flushleft}





\begin{flushleft}
maintain the pH values during 8.5--9.5. Phosphate existed
\end{flushleft}


\begin{flushleft}
in the form of HPO24 ? in the range of these pH values [26].
\end{flushleft}


\begin{flushleft}
Ca2 þ , OH ? and HPO24 ? formed a local condition with
\end{flushleft}


\begin{flushleft}
high concentration. This condition was beneficial to the
\end{flushleft}


\begin{flushleft}
formation of hydroxyapatite with pH ¼ 8.5--9.5.
\end{flushleft}


\begin{flushleft}
The mechanism could be further investigated by XRD.
\end{flushleft}


\begin{flushleft}
The XRD patterns of samples were compared (Fig. 8). The
\end{flushleft}


\begin{flushleft}
production was xonotlite (PDF card 23-0125, chemical
\end{flushleft}


\begin{flushleft}
formula Ca6Si6O17(OH)2) when Ca/Si molar ratio was 0.6:1
\end{flushleft}


\begin{flushleft}
and 1:1. For CSH: Ca/Si¼ 0.6, principal peaks of SiO2 were
\end{flushleft}


\begin{flushleft}
emerged at 20.3051 and 21.5621. The principal peaks in CSH:
\end{flushleft}


\begin{flushleft}
Ca/Si¼ 1.6 and CSH: Ca/Si¼ 2.2 were assigned to jennite
\end{flushleft}


\begin{flushleft}
(PDF card 18-1206; formula Ca9Si6O18(OH)6 ? 8 H2O; theoretical Ca/Si molar ratio of 1.5). The XRD pattern of CSH:
\end{flushleft}


\begin{flushleft}
Ca/Si¼ 2.2 showed the existence of Ca(OH)2. The coverage
\end{flushleft}


\begin{flushleft}
of the formed Ca(OH)2 was in good accordance with the
\end{flushleft}


\begin{flushleft}
result based on FESEM observations [27].
\end{flushleft}


\begin{flushleft}
The experiments indicated that jennite has the stronger
\end{flushleft}


\begin{flushleft}
capacity of Ca2 þ release compared with xonotlite and
\end{flushleft}


\begin{flushleft}
tobermorite. Low Ca/Si molar ratio resulted in the surplus
\end{flushleft}


\begin{flushleft}
of white carbon black. Therefore, a Si enriched layer was
\end{flushleft}


\begin{flushleft}
formed on the material's surface and blocked Ca2 þ
\end{flushleft}


\begin{flushleft}
release. Subsequently, phosphorus recovery capacity of
\end{flushleft}


\begin{flushleft}
material decreased. The formation of Ca(OH)2 was due
\end{flushleft}


\begin{flushleft}
to the surplus of carbide residue with high Ca/Si molar
\end{flushleft}


\begin{flushleft}
ratio. The single phosphorus removal efficiency of CSH:
\end{flushleft}


\begin{flushleft}
Ca/Si ¼ 2.2 was better than other samples due to the
\end{flushleft}


\begin{flushleft}
existence of Ca(OH)2. Nevertheless, massive Ca2 þ was
\end{flushleft}


\begin{flushleft}
released and reacted with phosphate ion as quickly as the
\end{flushleft}


\begin{flushleft}
material immersed in synthetic solution. Hydroxyapatite
\end{flushleft}


\begin{flushleft}
layer formed in a short time and led to obstruction of pore
\end{flushleft}


\begin{flushleft}
structure. So the capacity of Ca2 þ release decreased.
\end{flushleft}


\begin{flushleft}
4. Conclusions
\end{flushleft}


\begin{flushleft}
Porous calcium--silicate--hydrate was synthesized by carbide
\end{flushleft}


\begin{flushleft}
residue and white carbon black with a dynamic hydrothermal
\end{flushleft}


\begin{flushleft}
method. Ca/Si molar ratio exerted significant influence on
\end{flushleft}





\begin{flushleft}
phosphorus recovery performance of porous calcium--silicate--
\end{flushleft}


\begin{flushleft}
hydrate. The Ca/Si molar ratio of 1.6 for porous calcium--
\end{flushleft}


\begin{flushleft}
silicate--hydrate is more proper to recover phosphorus. Porous
\end{flushleft}


\begin{flushleft}
calcium--silicate--hydrate could recover phosphorus with phosphorus content of 18.64\%.
\end{flushleft}


\begin{flushleft}
The law of Ca2 þ and OH ? release was the key of
\end{flushleft}


\begin{flushleft}
phosphorus recovery efficiency. Changes of Ca/Si molar
\end{flushleft}


\begin{flushleft}
ratio led to the different pore structure. The increase of
\end{flushleft}


\begin{flushleft}
specific surface area and the increase in concentration of
\end{flushleft}


\begin{flushleft}
Ca2 þ release were in good agreement with each other.
\end{flushleft}


\begin{flushleft}
Further analysis by XRD indicated that two situations
\end{flushleft}


\begin{flushleft}
affected the law of Ca2 þ release. On the one hand, low Ca/Si
\end{flushleft}


\begin{flushleft}
molar ratio led to the formation of Si enriched layer. On the
\end{flushleft}


\begin{flushleft}
other hand, Ca(OH)2 formed due to high Ca/Si molar ratio.
\end{flushleft}


\begin{flushleft}
References
\end{flushleft}


\begin{flushleft}
[1] K. Suzkui, Y. Tanaka, K. Kuioda, K. Kuroda, D. Hanajima,
\end{flushleft}


\begin{flushleft}
Y. Fukumoto, T. kasuda, M. Waki, Removal and recovery of
\end{flushleft}


\begin{flushleft}
phosphorous from swine wastewater by demonstration crystallization reactor and struvite accumulation device, Bioresource Technology 8 (2007) 1573--1578.
\end{flushleft}


\begin{flushleft}
[2] Y.L. Yang, X. Li, C.X. Guo, F.W. Zhao, F. Jia, Efficiency and
\end{flushleft}


\begin{flushleft}
mechanism of phosphorus removal by coagulation of iron--manganese
\end{flushleft}


\begin{flushleft}
composited oxide, Chemical Research in Chinese Universities 2 (2009)
\end{flushleft}


224--227.


\begin{flushleft}
[3] Y.H. Song, D. Donnert, U. Berg, G.W. Peter, R. Nueesch, Seed
\end{flushleft}


\begin{flushleft}
selections for crystallization of calcium phosphate for phosphorus
\end{flushleft}


\begin{flushleft}
recovery, Journal of Environmental Science 19 (2007) 591--595.
\end{flushleft}


\begin{flushleft}
[4] X.C. Chen, H.N. Kong, D.Y. Wu, X.Z. Wang, Y.Y. Lin, Phosphate
\end{flushleft}


\begin{flushleft}
removal and recovery through crystallization of hydroxyapatite using
\end{flushleft}


\begin{flushleft}
xonotlite as seed crystal, Journal of Environmental Science 21 (2009)
\end{flushleft}


575--580.


\begin{flushleft}
[5] V.M. Elisabeth, K. Barr, Controlled struvite crystallisation for
\end{flushleft}


\begin{flushleft}
removing phosphorus from anaerobic digester sidestreams, Journal
\end{flushleft}


\begin{flushleft}
of Water Resources 1 (2001) 151--159.
\end{flushleft}


\begin{flushleft}
[6] Y.H. Song, G.W. Peter, U. Berg, R. Nueesch, D. Donnert, Calciteseed crystallization of calcium phosphate for phosphorus recovery,
\end{flushleft}


\begin{flushleft}
Chemosphere 63 (2006) 236--243.
\end{flushleft}


\begin{flushleft}
[7] J.B. Liu, X.Y. Ye, H. Wang, M.K. Zhu, B. Wang, H. Yan, The
\end{flushleft}


\begin{flushleft}
influence of pH and temperature on the morphology of hydroxyapatite synthesized by hydrothermal method, Ceramics International
\end{flushleft}


29 (2003) 629--633.


\begin{flushleft}
[8] C.R. Hood, A.A. Randall, A biochemical hypothesis explaining the
\end{flushleft}


\begin{flushleft}
response of enhanced biological phosphorus removal biomass to
\end{flushleft}


\begin{flushleft}
organic substrates, Water Research 11 (2011) 2758--2766.
\end{flushleft}


\begin{flushleft}
[9] P. Battistoni, P. Pavan, M. Prisciandaro, F. Cecchi, Struvite crystallization: a feasible and reliable way to fix phosphorus in anaerobic
\end{flushleft}


\begin{flushleft}
supernatants, Water Research 11 (2000) 3033--3041.
\end{flushleft}


\begin{flushleft}
[10] S. Sengupta, A. Pandit, Selective removal of phosphorus from
\end{flushleft}


\begin{flushleft}
wastewater combined with its recovery as a solid-phase fertilizer,
\end{flushleft}


\begin{flushleft}
Water Research 45 (2011) 3318--3330.
\end{flushleft}


\begin{flushleft}
[11] P. Battistoni, A. Dangelis, P. Pavan, M. Prisciandaro, F. Cecchi,
\end{flushleft}


\begin{flushleft}
Phosphorus removal from a real anaerobic supernatant by struvite
\end{flushleft}


\begin{flushleft}
crystallization, Water Research 9 (2001) 2167--2178.
\end{flushleft}


\begin{flushleft}
[12] A. Renman, G. Renman, Long-term phosphate removal by the
\end{flushleft}


\begin{flushleft}
calcium--silicate material Polonite in wastewater filtration systems,
\end{flushleft}


\begin{flushleft}
Chemosphere 79 (2010) 659--664.
\end{flushleft}


\begin{flushleft}
[13] L.E. de-Bashan, Y. Bashan, Recent advances in removing phosphorus from wastewater and its future use as fertilizer (1997--2003),
\end{flushleft}


\begin{flushleft}
Water Research 38 (2004) 4222--4246.
\end{flushleft}


\begin{flushleft}
[14] H.B. Yin, Y. Yun, Y.L. Zhang, C.X. Fan, Phosphate removal from
\end{flushleft}


\begin{flushleft}
wastewaters by a naturally occurring, calcium-rich sepiolite, Journal
\end{flushleft}


\begin{flushleft}
of Hazardous Materials 198 (2011) 362--369.
\end{flushleft}





\begin{flushleft}
\newpage
W. Guan et al. / Ceramics International 39 (2013) 1385--1391
\end{flushleft}


\begin{flushleft}
[15] L.J. Westholm, Substrates for phosphorus removal---potential benefits for on-site wastewater treatment?, Water Research 40 (2006)
\end{flushleft}


23--36


\begin{flushleft}
[16] L. Baur, K. Peter, D. Mavrocordatos, B. Wehrli, C.A. Johnson,
\end{flushleft}


\begin{flushleft}
Dissolution--precipitation behaviour of ettringite, monosulfate, and
\end{flushleft}


\begin{flushleft}
calcium silicate hydrate, Cement and Concrete Research 34 (2004)
\end{flushleft}


341--348.


\begin{flushleft}
[17] J.C. Jeffrey, J.J. Thomas, Hal F.W. Taylor, H.M. Jennings, Solubility and structure of calcium silicate hydrate, Cement and Concrete
\end{flushleft}


\begin{flushleft}
Research 34 (2004) 1499--1519.
\end{flushleft}


\begin{flushleft}
[18] S.U. Sezen, Sejung S.R. Chae, C.J. Benmore, H.R. Wenk, Compositional evolution of calcium silicate hydrate(C--S--H) structures by
\end{flushleft}


\begin{flushleft}
total X-ray scattering, Journal of the American Ceramic Society 2
\end{flushleft}


(2012) 793--798.


\begin{flushleft}
[19] I.G. Richardson, Tobermorite/jennite-and tobermorite/calcium
\end{flushleft}


\begin{flushleft}
hydroxide-based models for the structure of C--S--H: applicability
\end{flushleft}


\begin{flushleft}
to hardened pastes of tricalcium silicate, b-dicalcium silicate, Portland cement, and blends of Portland cement with blast-furnace slag,
\end{flushleft}


\begin{flushleft}
metakaolin, or silica fume, Cement and Concrete Research 34 (2004)
\end{flushleft}


1733--1777.


\begin{flushleft}
[20] M.Q. Li, H.X. Liang, Formation of micro-porous spherical particles
\end{flushleft}


\begin{flushleft}
of calcium silicate (xonotlite) in dynamic hydrothermal process,
\end{flushleft}


\begin{flushleft}
China Particuology 3 (2004) 124--127.
\end{flushleft}





1391





\begin{flushleft}
[21] A.A.P. Mansur, H.S. Mansur, Preparation, characterization and
\end{flushleft}


\begin{flushleft}
cytocompatibility of bioactive coatings on porous calcium--silicate--
\end{flushleft}


\begin{flushleft}
hydrate scaffolds, Materials Science and Engineering C 30 (2010)
\end{flushleft}


288--294.


\begin{flushleft}
[22] J.P. Gustafsson, A. Renman, G. Renman, K. Poll, Phosphate
\end{flushleft}


\begin{flushleft}
removal by mineral-based sorbents used in filters for small-scale
\end{flushleft}


\begin{flushleft}
wastewater treatment, Water Research 42 (2008) 189--197.
\end{flushleft}


\begin{flushleft}
[23] J. Kim, C. Vipulanandan, Effect of pH, sulfate and sodium on the
\end{flushleft}


\begin{flushleft}
EDTA titration of calcium, Cement and Concrete Research 33 (2003)
\end{flushleft}


621--627.


\begin{flushleft}
[24] W.S. Zhang, H.X. Wang, J.Y. Ye, Pore structure and surface fractal
\end{flushleft}


\begin{flushleft}
characteristics of calcium silicate hydrates contained organic macromolecule, Journal of the Chinese Ceramic Society 12 (2006)
\end{flushleft}


\begin{flushleft}
1497--1502 (in Chinese).
\end{flushleft}


\begin{flushleft}
[25] N. Demirkiran, A. Kunkul, Dissolution kinetics of ulexite in
\end{flushleft}


\begin{flushleft}
perchloric acid solutions, International Journal of Mineral Processing 83 (2007) 76--80.
\end{flushleft}


\begin{flushleft}
[26] Y. Liu, X. Sheng, Y.H. Dong, Y.J. Ma, Removal of high-concentration phosphate by calcite: effect of sulfate and pH, Desalination 289
\end{flushleft}


(2012) 66--71.


\begin{flushleft}
[27] E. Gallucci, K. Scrivener, Crystallization of calcium hydroxide in
\end{flushleft}


\begin{flushleft}
early age model and ordinary cementitious systems, Cement and
\end{flushleft}


\begin{flushleft}
Concrete Research 37 (2007) 492--501.
\end{flushleft}





\newpage



\end{document}
